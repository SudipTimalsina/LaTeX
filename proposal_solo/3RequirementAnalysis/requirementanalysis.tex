\chapter{REQUIREMENT ANALYSIS}
\section{Functional Requirements}
Functional requirements are product features or functions that developers must implement to enable users to accomplish their tasks. Generally, functional requirements describe system behavior under specific conditions. The functional requirements for our project are given below:
\begin{itemize}[itemsep=-4pt, topsep=-8pt]
    \item User should be able to register as the donor or requester at any time
    \item User should be able to receive the notification sent by the app itself.
    \item Users should be able to view their previous donations history.
    \item User should be able to share donation history or anything to other social platform.
\end{itemize}

\section{Non-Functional Requirements}
Non-functional requirements are a set of specifications that describe the system's operation capabilities and constraints and attempt to improve its functionality. The non-functional requirements required for our project are given below:
\begin{itemize}[itemsep=-4pt, topsep=-8pt]
\item The applications should be user-friendly and intuitive for ease of use. 
\item The application should perform with efficient throughput and response time.
\item The system should be scalable and should accommodate a growing user base.
\item The project should provide a tamper proof and secure digital identity storage.
\item The system should be able to handle all types of identity documents submitted by the user.

% \item The project should provide a high quality digital identity management system.
\end{itemize}


\section{Feasibility Study}
\subsection{Economic Feasibility}
To determine the economic feasibility of this project, the cost associated with developing and launching the application must be evaluated. The costs associated with this project are mostly related to costs of developing the project with the utmost security and cost of maintaining the server. This project reduces the cost and time of searching the doners at the time of need and also provides the information about the nearest ongoing donation program.

\subsection{Operational Feasibility}
The system proposed has mobile application and web application with helpful user interfaces, so the end-users wouldn't have problem operating the required application with ease. The widespread use of smartphones and increasing digital literacy among users make the app accessible to a large audience. With its user-friendly interface and real-time communication capabilities, the app can seamlessly integrate into users' daily lives, enabling them to quickly respond to urgent blood donation requests.

\subsection{Technical Feasibility}
Creating the Blood Donating App is possible because we have tools to build mobile apps, use cloud technology, and connect in real-time. We can accurately show nearby blood donation centers using maps, and make sure data is safe with encryption. Skilled developers and existing resources make it doable and effective.

% The titled project “Parichaya” has the potential to provide various benefits like increased security, convenience and cost savings. Usage of “ Hyperledger Fabric” to store digital identity documents makes the documents tamper proof and immutable increasing security of the documents. The project is scalable and can accommodate a growing user base. The application is user-friendly and intuitive and provides secure digital identity storage. Hence, the project if properly executed can provide a high quality digital identity management system.




\section{Software Requirements}

\textbf{Figma} \newline
Figma is a collaborative web application for interface design, with additional offline features enabled by desktop applications for macOS and Windows. The feature set of Figma focuses on user interface and user experience design, with an emphasis on real-time collaboration,[1] utilising a variety of vector graphics editor and prototyping tools. The Figma mobile app for Android and iOS allows viewing and interacting with Figma prototypes in real-time on mobile and tablet devices.

\textbf{Express js}\newline
Expressing the essence of efficient web application development, Express.js stands as a versatile and streamlined framework within the Node.js ecosystem. As a minimalist and flexible tool, Express.js empowers developers to construct powerful web applications and APIs with ease. Its core functionality lies in simplifying the handling of HTTP requests, routes, and middle ware. By offering a clean and organized approach to creating restful services, Express.js enables developers to swiftly define routes for diverse URLs and HTTP methods. It also supports middle ware integration for tasks like authentication and logging, ensuring a seamless request-response cycle. While not including a built-in template engine, Express.js readily adapts to popular engines such as EJS and Pug for dynamic content generation.
% \end{itemize}

% \begin{itemize}
\textbf{React JS}\newline
   Our project involves the issuance of sensitive documents like citizenship, national identity, and driving licenses through a government portal, using ReactJS can be a valuable tool for developing the front-end interface of our application. ReactJS is a popular front-end framework that allows developers to build dynamic and responsive user interfaces. With ReactJS, developers can create reusable components and build an interactive user interface that allows authorized persons of the government to issue sensitive documents securely. ReactJS also has state management features that make it easier to handle complex user interactions and manage the application's data flow.
   
\textbf{Node JS}\newline
Node.js serves as the runtime environment for the backend, allowing us to use JavaScript on the server-side. This enables consistent code throughout the stack and leverages the extensive Node.js package ecosystem, enhancing development efficiency.

 \textbf{MongoDB}\newline
 MongoDB is a popular open-source, NoSQL database management system that stores data in a flexible, document-oriented format. It's designed to handle large volumes of unstructured or semi-structured data, making it suitable for various applications like web applications, mobile apps, and more. In MongoDB, data is stored in collections, which are similar to tables in traditional relational databases. Each data entry is represented as a JSON-like document, which can have varying structures within the same collection. This schema flexibility allows for easier adaptation to changing data requirements. Key features of MongoDB include its ability to scale horizontally across multiple servers, automatic sharding for distributing data across nodes, and support for geospatial queries and indexing.
