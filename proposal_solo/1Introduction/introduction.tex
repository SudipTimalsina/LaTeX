\chapter{INTRODUCTION}
% (20% of Proposal Length)
\pagenumbering{arabic}




\section{Background}
Blood donation is an essential component of healthcare systems around the world, serving as a lifeline for individuals enduring operations, medical treatments, and crises. However, the ongoing issues of blood scarcity, timely access, and efficient coordination have highlighted the need for novel solutions. Traditional blood donation techniques sometimes rely on manual processes and disconnected communication channels, resulting in inefficiencies and missed possibilities for life-saving donations.

In response to these problems, the proposed Blood Donation App emerges as a game changer. This software takes advantage of mobile technology's pervasiveness to create a dynamic and integrated network that connects donors, recipients, and healthcare organizations. The app seeks to alter the landscape of blood donation and distribution by providing real-time communication, faster processes, and instructional tools. The app envisions a future in which every potential donor may easily donate, every recipient can quickly access the blood they require, and every healthcare professional can successfully handle donation-related operations, thanks to its user-friendly interface and comprehensive capabilities.

\section{Motivation}
The Blood Donating App was inspired by the urgent need to revolutionize blood donation systems. In a world filled with medical emergencies, surgeries, and healthcare challenges, the app stands out as a glimmer of hope, motivated by the desire to overcome the limitations of traditional donation methods. The app aims to seamlessly connect donors, recipients, and healthcare institutions by leveraging the power of mobile technology and real-time connectivity, transforming blood donation into a streamlined and efficient process. This attempt is driven by a deep desire to save lives and address the persistent issues of blood shortages and ineffective communication.The app's aim is to rise to the existing hurdles by providing a platform that crosses physical borders and allows users to simply make life-saving donations.

\section{Problem Statement}
The way we donate and distribute blood needs improvement. It's often slow, and sometimes there's not enough blood when people urgently need it. Also, not many people know about donating blood or how important it is. COVID-19 has shown we need a better system for emergencies too. We need a solution that connects donors, recipients, and hospitals quickly. That's where the Blood Donating App comes in. It aims to make donating blood easy, educate people, and help during emergencies. By using this app, we can solve these problems and save lives by giving blood to those who need it. 

\section{Objectives}

 \begin{itemize}

    \item Establish a real-time platform connecting blood donors and recipients to ensure efficient and timely blood distribution.
   
\end{itemize}

% \section{Assumptions}
% \begin{itemize}
%     \item The document issued by Government Organizations is authenticated and valid.
%     \item Each citizen  has national identity number already allocated to them and every other identity documents is associated with it.
% \end{itemize}


\section{Scope of Project}

The goal of the Blood Donating App is to create a user-friendly platform that enables administrators, receivers, and donors to easily register for, request, and manage blood donations. Donor replies to urgent requests will be ensured in real time. The software will act as a teaching tool, dispelling misconceptions and promoting awareness of blood donation. Donors may track their progress and rewards, and the incorporation of social media will boost overall engagement. Additionally, improved coordination between blood banks, donors, recipients, and healthcare organizations will boost the process' overall effectiveness, helping to save lives and encourage voluntary blood donation.


